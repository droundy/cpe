\documentclass[twocolumn]{article}

\usepackage{amsmath}
\usepackage{graphicx}

\title{Electrolytic Constant Phase Elements}
\author{David Roundy}

\begin{document}

\maketitle

Here are some nice plots.  We should really describe them and write
some text.

\begin{figure}
  \includegraphics[width=\columnwidth]{potential-energy-vs-distance}
  \caption{This is the potential energy computed using an assumption
    of simple electrostatics with a continuum dielectric.  We expect
    this to break down at short distances from the graphene, distances
    comparable to the important solvation shells.}
  \label{fig:pot}
\end{figure}

\begin{figure}
  \includegraphics[width=\columnwidth]{concentration-vs-distance}
  \caption{This is the concentration of ions versus distance from the
    graphene, based on the potential in Fig.~\ref{fig:pot}.}
  \label{fig:concentration}
\end{figure}

\begin{figure}
  \includegraphics[width=\columnwidth]{resistivity-vs-distance}
  \caption{This is the local resistivity, based on the concentration
    in Fig.~\ref{fig:concentration}.}
\end{figure}

\begin{figure}
  \includegraphics[width=\columnwidth]{impedance-vs-frequency}
  \caption{Finally, this is the impedance as a function of frequency
    based on theory (solid lines, see also previous figures).  Also
    plotted are the experimental values for several graphene sheets.}
\end{figure}

\clearpage

\appendix

\section{Concentration}

The concentration as a function of DC voltage using Poisson-Boltzmann
equation.  I will use Gaussian units, which means that Poisson's
equation is given by:
\begin{align}
  \nabla^2 \psi = \frac{4\pi}{\epsilon} \rho
\end{align}
For simplicity we assume that we have single-charged cations and
anions.
\begin{align}
  \rho = e (n_{+} - n_{-})
\end{align}
We can find the number densities of ions from
\begin{align}
  n_{\pm} = n_0 \exp(\mp\beta e \psi)
\end{align}
keeping in mind that $\psi\rightarrow 0$ far from the surface.

Actually, we can solve for $\psi(x)$ analytically, and from that
obtain the densities.

\begin{figure}
  \includegraphics[width=\columnwidth]{poisson}
\end{figure}

\begin{figure}
  \includegraphics[width=\columnwidth]{sigma-vs-V}
\end{figure}

\section{Conductivity}

\begin{figure}
  \includegraphics[width=\columnwidth]{resistivity}
  \caption{Resistivity as a function of distance from the capacitor
    surface.}
\end{figure}

The conductivity is a function of ion concentration.  At sufficiently
low densities, the conductivity due to each species is proportional to
its density.  At higher densities, we should use Kohlrausch's law,
which accounts for repulsive interactions between ions of the same
charge.  Sadly, Kohlrausch's law fails at very high concentrations,
which is a problem because the concentration tends to get very high
near the charged surface.

\end{document}
