\documentclass[twocolumn]{revtex4-1}

\usepackage{amsmath}
\usepackage{graphicx}
\usepackage{afterpage}
\usepackage{color}

\newcommand\fixme[1]{{\bf\color{red}[FIX ME! #1 !EM XIF]}}

\begin{document}

\title{Electrolytic Constant Phase Elements}
\author{MacKenzie Lenz}
\author{David Roundy}
\affiliation{Department of Physics, Oregon State University, Corvallis, OR 97331}


\maketitle

Here are some nice plots.  We should really describe them and write
some text.  The theory comes from
Hirschorn~\cite{hirschorn2010constant}.

\begin{figure}
  \includegraphics[width=\columnwidth]{potential-energy-vs-distance}
  \caption{This is the potential energy computed using an assumption
    of simple electrostatics with a continuum dielectric.  We expect
    this to break down at short distances from the graphene, distances
    comparable to the important solvation shells.}
  \label{fig:pot}
\end{figure}

\begin{figure}
  \includegraphics[width=\columnwidth]{concentration-vs-distance}
  \caption{This is the concentration of ions versus distance from the
    graphene, based on the potential in Fig.~\ref{fig:pot}.}
  \label{fig:concentration}
\end{figure}

\begin{figure}
  \includegraphics[width=\columnwidth]{resistivity-vs-distance}
  \caption{This is the local resistivity, based on the concentration
    in Fig.~\ref{fig:concentration}.}
\end{figure}

\begin{figure}
  \includegraphics[width=\columnwidth]{impedance-vs-frequency}
  \caption{Finally, this is the impedance as a function of frequency
    based on theory (solid lines, see also previous figures).  Also
    plotted are the experimental values for several graphene sheets.}
\end{figure}

\afterpage{\clearpage}

\clearpage

\section{Finding the capacitance from the potential energy}

How does concentration come from potential energy?

\begin{align}
  n_i(x) = n_{i0} \exp\left(-\beta \left(V_{\text{graphene}}(x)
                                       + q_i \Phi(x)\right)\right)
\end{align}
where $\Phi$ is the electrostatic potential.  Then we can compute the
charge density $\rho(x)$ from
\begin{align}
  \rho(x) = \sum_i q_i n_i(x)
\end{align}
and finally, the Poisson equation...
\begin{align}
  \nabla^2 \Phi(x) = -\frac{\rho(x)}{\epsilon_0\epsilon_w}
\end{align}
\begin{align}
  \nabla^2 \Phi(x) = -\frac{\sum_i q_i n_i(x)}{\epsilon_0\epsilon_w}
\end{align}
Gauss' Law!


\section{Diffusion and Warburg impedance}

Another possibility to consider is that the current response may be
diffusion-limited.  Relevant to this is the fact that the diffusion
constant for NaCl is around 1.5$\times 10^{-6}$ cm$^2$/s.  Given our
highest frequency of $10^4$ Hz, we can expect a diffusion distance of
around about $10^{-5}$ cm, which is $100$~nm.  So I don't \emph{think}
we should see Warburg impedance.


\clearpage

\section{Finding potential energy from impedance}

\begin{figure}
  \includegraphics[width=\columnwidth]{optimal-Z-fit}
  \caption{The frequency-dependent impedance, experiment
    vs. theoretical fit.}\label{fig:optimal-Z-fit}
\end{figure}

\begin{figure}
  \includegraphics[width=\columnwidth]{optimal-resistivity-and-concentration}
  \caption{The resistivity and concentration curves that best fits the
    impedance of one particular
    experiment.}\label{fig:optimal-resistivity-and-concentration}
\end{figure}

\begin{figure}
  \includegraphics[width=\columnwidth]{optimal-potential}
  \caption{The potential energy curve extracted from the
    concentration, which is iteslf extracted from the resistivity that
    we fit to the impedance.  The horizontal dashed line is the
    solvation free energy of the sodium cation as cited in
    Ref.~\cite{horinek2009rational}.}\label{fig:optimal-potential}
\end{figure}

I have written some code\footnote{see \texttt{transposed\_resistivity.py}} that
finds the resistivity curve that best fits the experimental
impedence.  See Fig.~\ref{fig:optimal-Z-fit} for a plot of the fit.
One thing I discovered was that I seem to have the real and imaginary
parts swapped in either experiment or theory.  Or at least, I'm unable
to reproduce experiment without swapping the two.  This should be
looked into.

Figure~\ref{fig:optimal-resistivity-and-concentration} shows the
optimal resistivity curve.  The fit works by adjusting the $\Delta x$
values (actually, $\sqrt{\Delta x}$) and thus $x(\rho)$, rather than
$\rho(x)$.  This turns the function from a nonlinear integration into
a linear summation of the varying quantities, which seems simpler.

As can be seen in Fig.~\ref{fig:optimal-potential}, we end up with an
almost linear potential energy, which seems rather funky and
surprising.  This curve also suggests that the hydration energy of the
ions is almost 12~eV, which seems (purely as a gut feeling) a bit on
the high side.

\clearpage

\appendix

\section{Concentration}

The concentration as a function of DC voltage using Poisson-Boltzmann
equation.  I will use Gaussian units, which means that Poisson's
equation is given by:
\begin{align}
  \nabla^2 \psi = \frac{4\pi}{\epsilon} \rho
\end{align}
For simplicity we assume that we have single-charged cations and
anions.
\begin{align}
  \rho = e (n_{+} - n_{-})
\end{align}
We can find the number densities of ions from
\begin{align}
  n_{\pm} = n_0 \exp(\mp\beta e \psi)
\end{align}
keeping in mind that $\psi\rightarrow 0$ far from the surface.

Actually, we can solve for $\psi(x)$ analytically, and from that
obtain the densities.

\begin{figure}
  \includegraphics[width=\columnwidth]{poisson}
\end{figure}

\begin{figure}
  \includegraphics[width=\columnwidth]{sigma-vs-V}
\end{figure}

\section{Conductivity}

\begin{figure}
  \includegraphics[width=\columnwidth]{resistivity}
  \caption{Resistivity as a function of distance from the capacitor
    surface.}
\end{figure}

The conductivity is a function of ion concentration.  At sufficiently
low densities, the conductivity due to each species is proportional to
its density.  At higher densities, we should use Kohlrausch's law,
which accounts for repulsive interactions between ions of the same
charge.  Sadly, Kohlrausch's law fails at very high concentrations,
which is a problem because the concentration tends to get very high
near the charged surface.

\section{Hydration properties of ions}
Here I will stick a little lit review of properties of ions in water.

Mancinelli \emph{et al.} measured the radial distribution of water (O
and H both) around Na$^+$, K$^+$, and Cl$^-$ ions at different
concentrations~\cite{mancinelli2007hydration}.  They find that the
first hydration shell is 2 or 3 \AA\ from the cation or anion
respectively.  This paper should be our go-to paper for the hydration
of these ions.

Tielrooij \emph{et al.} used THz spectroscopy to examine the dynamics
of water around ions~\cite{tielrooij2010cooperativity}.  They conclude
that the ions can significantly affect water molecules well beyond the
first solvation layer.  Moreover, they found that these effects were
highly dependent on nature of the counterion, and were nonadditive.

M\"ahler \emph{et al.} studied experimentally the hydration properties
of the alkali metal ions~\cite{mahler2011study}.  They don't seem to
have much that Ref.~\citenum{mancinelli2007hydration} does not, but
better to have two experiments than one!

Horinek \emph{et al.} wrote a theoretical paper designing empirical
potentials for ions based on thermodynamic data, specifically the free
energy and entropy of solvation~\cite{horinek2009rational}.  They
discuss the experimental difficulty in measuring single-ion solvation
data, and choose to set chloride as a reference.  In
Fig.~\ref{fig:optimal-potential}, I plot a red line based on their
estimated solvation free energy for the sodium cation.  If we include
or discuss this in our paper, we should reread this paper carefully,
and see if we can find some appropriate experimental data to compare
with and cite.


\bibstyle{plain}
\bibliography{cpe}

\end{document}
