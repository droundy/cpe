\documentclass[twocolumn]{revtex4-1}

\usepackage{amsmath}
\usepackage{graphicx}
\usepackage{color}

\newcommand\fixme[1]{{\bf\color{red}[FIX ME! #1 !EM XIF]}}

\begin{document}

\title{Analysis of Morgan's Data}
\author{David Roundy}
\affiliation{Department of Physics, Oregon State University, Corvallis, OR 97331}

\maketitle

\section{Simple analysis}

\begin{figure}
  \includegraphics[width=\columnwidth]{hall-measurements/vh-vs-vg}
  \includegraphics[width=\columnwidth]{hall-measurements/vxx-vs-vg}
  \caption{Raw Hall data.}
  \label{fig:raw-data}
\end{figure}

\begin{figure}
  \includegraphics[width=\columnwidth]{hall-measurements/ns-vs-vg}
  \includegraphics[width=\columnwidth]{hall-measurements/mu-vs-vg}
  \caption{Simple analysis of Hall data.}
  \label{fig:simple-analysis}
\end{figure}

We begin with the raw data.  Fig.~\ref{fig:raw-data} shows the raw
data that was collected with a B field of 0.5~Tesla, and with a
current of $1\times 10^{-6}$ amps.

The traditional analysis uses the above data to extract two
quantities: the signed carrier density and the mobility.  The carrier
density comes from
\begin{align}
  n_s &= \frac{BI}{V_He}\label{eq:ns}
\end{align}
while the mobility is obtained from
\begin{align}
  \mu &= \frac{V_H}{V_{xx}B}\label{eq:mu}
\end{align}
The output of this analysis is shown in
Fig.~\ref{fig:simple-analysis}.  The approach misbehaves where the
Hall voltage approaches zero (at gate voltages less than about 0.2 V),
since the resistivity ought to diverge (at zero carrier density), but
it doesn't because graphene on a substrate can hold both positive and
negative carriers due to electrostatic inhomogeneity.

There are two main challenges with this analysis.  One is the
aforementioned issue that the carriers do not all have the same sign.
The second is that the gate voltage is not trivially related to the
electrostatic voltage across the double layer capacitor.

\section{Including both carriers}
If we incorporate both carriers into the analysis, we introduce an
additional unknown density, which means we need an additional
assumption in order to solve for the two carrier densities.  One such
assumption would be that the mobility is independent of gate voltage,
and that the electron and hole mobilities are identical.  These seem
like reasonable guesses, given what we know about the band structure
of graphene.

We begin with an assumption of equal mobility for both carriers so
that in the absence of a magnetic field we will have:
\begin{align}
  \vec{v}_+ &= \mu \vec{E} \\
  \vec{v}_- &= -\mu \vec{E} \\
  \vec{J} &= e n_+ \vec{v}_+ - e n_- \vec{v}_- \\
  &= \mu\left(n_+ + n_-\right) e \vec{E}
\end{align}
We continue with the ordinary Hall measurement idea that there is a
vertical magnetic field and zero transverse current.  Further, we
assume that a steady state is reached such that the mean Lorentz force
on each carrier type is zero.
\begin{align}
  \vec{F}_\pm &= \pm e(\vec{E} + \vec{v}_\pm\times\vec{B}) \\
  \vec{v}_\pm &= \pm \mu (\vec{E} + \vec{v}_\pm\times\vec{B}) \\
  v_{\pm\parallel} &= \pm \mu (E_\parallel + v_{\pm\perp}B) \\
  v_{\pm\perp} &= \pm \mu (E_\perp - v_{\pm\parallel}B) \\
  &= \pm \mu (E_\perp \mp \mu (E_\parallel + v_{\pm\perp}B)B) \\
  &= \pm \mu (E_\perp \mp \mu E_\parallel B \mp \mu v_{\pm\perp}B^2) \\
  &\approx \pm \mu (E_\perp \mp \mu E_\parallel B) \\
  &= \pm \mu E_\perp - \mu^2 E_\parallel B
\end{align}
Now that we know how fast the charges are moving, given the electric
and magnetic fields they feel, we can apply the constraint that there
is no transverse current.
\begin{align}
  J_\perp
  &= en_+v_{+\perp} - en_-v_{-\perp} \\
  &= e(n_+v_{+\perp} - n_-v_{-\perp}) \\
  &= e\mu\left(n_+(E_\perp - \mu E_\parallel B)
  - n_-(-E_\perp - \mu E_\parallel B) \right) \\
             &= e\mu\left( (n_+ + n_-) E_\perp
             - (n_+ - n_-)\mu E_\parallel B \right) \\
             &= 0
\end{align}
Thus we obtain a relationship between the parallel and transverse
fields, as well as the densities.
\begin{align}
  (n_+ + n_-) E_\perp &= (n_+ - n_-)\mu E_\parallel B
\end{align}
At this point it is probably worth thinking about how we can connect
with the measured quantities.  Let us begin by imagining that our
system has length $\ell$ and width $w$, and is of rectangular shape.
First, we will consider finding potentials from electric fields, so we
can connect those will experiment also.
\begin{align}
  V_\perp &= w E_\perp \\
  V_\parallel &= \ell E_\parallel
\end{align}
Now let us ask about the current, which is one of our other
experimental variables.
\begin{align}
  I &= J_\parallel w \\
  &= (en_+v_{+\parallel} - e n_-v_{-\parallel})w \\
  &\approx \mu e (n_+ + n_-) E_\parallel w \\
  I &= \mu e (n_+ + n_-) V_\parallel \frac{w}{\ell}
\end{align}
Now revisiting the transverse direction, we find:
\begin{align}
  (n_+ + n_-) \frac{V_\perp}{w} &= \mu (n_+ - n_-)\frac{V_\parallel}{\ell} B \\
  (n_+ + n_-) V_\perp &= \frac{n_+ - n_-}{n_+ + n_-}\frac{IB}{e} \\
  \frac{(n_+ + n_-)^2}{n_+ - n_-} &= \frac{IB}{V_\perp e}
\end{align}
This is the clarification of Eq.~\ref{eq:ns} which gave $n_s$ in the
simple interpretation of the Hall measurement.  We can see how it
reduces to that formula if we set one of the two densities to zero.

Equation~\ref{eq:mu} was obtained by finding the ratio between the two
voltages.  In our more complicated case, we find
\begin{align}
  \mu &= \frac{n_+ - n_-}{n_+ + n_-}\frac{V_\perp}{V_\parallel B}\frac{\ell}{w}
\end{align}
In this case, we can see that the formula reduces to the simpler one
if we set one of the densities to zero \emph{and} set the length and
width equal.

We have now obtained two equations involving our four experimental
parameters (plus $\ell$ and $w$, which we will assume may be measured,
or perhaps taken to be equal) and now three unknowns:  $\mu$, $n_+$,
and $n_-$.  To simplify the formulas, we will express them in terms of
two combinations of densities:
\begin{figure}
  \includegraphics[width=\columnwidth]{hall-measurements/xi-vs-vg}
  \includegraphics[width=\columnwidth]{hall-measurements/new-mu-vs-vg}
  \includegraphics[width=\columnwidth]{hall-measurements/nt-vs-vg}
  \caption{Fancy analysis.  The total carrier density is shown with
    solid lines.  Dashed lines show the positive carrier density,
    while dotted lines show the negative carrier density.}
  \label{fig:nice}
\end{figure}
\begin{figure}
  \includegraphics[width=\columnwidth]{hall-measurements/xiauto-vs-vg}
  \includegraphics[width=\columnwidth]{hall-measurements/muauto-vs-vg}
  \includegraphics[width=\columnwidth]{hall-measurements/ntauto-vs-vg}
  \caption{Fancy auto analysis.  The total carrier density is shown with
    solid lines.  Dashed lines show the positive carrier density,
    while dotted lines show the negative carrier density.}
  \label{fig:nice}
\end{figure}
\begin{align}
  n_t &\equiv n_+ + n_- \\
  \xi &\equiv \frac{n_+ - n_-}{n_+ + n_-}
\end{align}
Thus our two essential equations become:
\begin{align}
  n_t &= \xi\frac{IB}{V_\perp e} \\
  \xi  &= \frac{V_\perp}{\mu V_\parallel B}\frac{l}{w}
\end{align}
Now, we should be able to assume that for graphene $\mu$ is
relatively independent of gate voltage, due to the symmetry between
electrons and holes near the Dirac point.  Moreover, we should also be
able to assume that far from the Dirac point the minority fraction
should be zero, meaning that $\xi \rightarrow 1$.  Thus at large gate
voltages we should be able to solve for $\mu$ experimentally.  Then
given $\mu$ we should be able to solve for both $n_t$ and $\xi$.

Results are shown in Fig.~\ref{fig:nice}, which are disappointing.  We
expect $\xi$ to monotonically approach $\pm 1$, but find that it peaks
in both cases.  It also increases above a magnitude of 1 which is bad,
but arises only because I set the value of $\sigma$ based on the
largest gate voltages (approximately).

On the plus side, the total density is now more nicely behaved at
small gate voltages, so that is good.

\end{document}
